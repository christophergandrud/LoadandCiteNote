\documentclass[a4paper]{report}\usepackage{graphicx, color}
%% maxwidth is the original width if it is less than linewidth
%% otherwise use linewidth (to make sure the graphics do not exceed the margin)
\makeatletter
\def\maxwidth{ %
  \ifdim\Gin@nat@width>\linewidth
    \linewidth
  \else
    \Gin@nat@width
  \fi
}
\makeatother

\definecolor{fgcolor}{rgb}{0.2, 0.2, 0.2}
\newcommand{\hlnumber}[1]{\textcolor[rgb]{0,0,0}{#1}}%
\newcommand{\hlfunctioncall}[1]{\textcolor[rgb]{0.501960784313725,0,0.329411764705882}{\textbf{#1}}}%
\newcommand{\hlstring}[1]{\textcolor[rgb]{0.6,0.6,1}{#1}}%
\newcommand{\hlkeyword}[1]{\textcolor[rgb]{0,0,0}{\textbf{#1}}}%
\newcommand{\hlargument}[1]{\textcolor[rgb]{0.690196078431373,0.250980392156863,0.0196078431372549}{#1}}%
\newcommand{\hlcomment}[1]{\textcolor[rgb]{0.180392156862745,0.6,0.341176470588235}{#1}}%
\newcommand{\hlroxygencomment}[1]{\textcolor[rgb]{0.43921568627451,0.47843137254902,0.701960784313725}{#1}}%
\newcommand{\hlformalargs}[1]{\textcolor[rgb]{0.690196078431373,0.250980392156863,0.0196078431372549}{#1}}%
\newcommand{\hleqformalargs}[1]{\textcolor[rgb]{0.690196078431373,0.250980392156863,0.0196078431372549}{#1}}%
\newcommand{\hlassignement}[1]{\textcolor[rgb]{0,0,0}{\textbf{#1}}}%
\newcommand{\hlpackage}[1]{\textcolor[rgb]{0.588235294117647,0.709803921568627,0.145098039215686}{#1}}%
\newcommand{\hlslot}[1]{\textit{#1}}%
\newcommand{\hlsymbol}[1]{\textcolor[rgb]{0,0,0}{#1}}%
\newcommand{\hlprompt}[1]{\textcolor[rgb]{0.2,0.2,0.2}{#1}}%

\usepackage{framed}
\makeatletter
\newenvironment{kframe}{%
 \def\at@end@of@kframe{}%
 \ifinner\ifhmode%
  \def\at@end@of@kframe{\end{minipage}}%
  \begin{minipage}{\columnwidth}%
 \fi\fi%
 \def\FrameCommand##1{\hskip\@totalleftmargin \hskip-\fboxsep
 \colorbox{shadecolor}{##1}\hskip-\fboxsep
     % There is no \\@totalrightmargin, so:
     \hskip-\linewidth \hskip-\@totalleftmargin \hskip\columnwidth}%
 \MakeFramed {\advance\hsize-\width
   \@totalleftmargin\z@ \linewidth\hsize
   \@setminipage}}%
 {\par\unskip\endMakeFramed%
 \at@end@of@kframe}
\makeatother

\definecolor{shadecolor}{rgb}{.97, .97, .97}
\definecolor{messagecolor}{rgb}{0, 0, 0}
\definecolor{warningcolor}{rgb}{1, 0, 1}
\definecolor{errorcolor}{rgb}{1, 0, 0}
\newenvironment{knitrout}{}{} % an empty environment to be redefined in TeX

\usepackage{alltt}
\usepackage[utf8]{inputenc}
\usepackage[T1]{fontenc}
\usepackage{RJournal}
\usepackage[round]{natbib}
\bibliographystyle{abbrvnat}
\usepackage{amsmath,amssymb,array}
\usepackage{booktabs}

%% load any required packages here
\IfFileExists{upquote.sty}{\usepackage{upquote}}{}

\begin{document}

%% do not edit, for illustration only
\fancyhf{}
\fancyhead[LO,RE]{\textsc{Contributed Article}}
\fancyhead[RO,LE]{\thepage}
\fancyfoot[L]{The R Journal Vol. X/Y, Month, Year}
\fancyfoot[R]{ISSN 2073-4859}

\begin{article}

\title{Automating R Package Citations in Reproducible Research Documents}
\author{Christopher Gandrud}

\maketitle

\abstract{
  It is clearly best practice to fully document all of the R packages a reproducible computational document relies. Because the syntax and capabilities of an R package change over time, without a clear knowledge of what version of package was used, it could be difficult to actually reproduce a piece of research. In this short note, I introduce the \code{LoadandCite} command from the \pkg{repmis} package. If included in a reproducible research document it can (a) load all the R packages used to create the document, (b) create a BibTeX file with the full citation information of each package loaded and updates the file every time the paper is compiled. It also allows the user to automatically install all of the packages and even specific versions of the packages. This will make it much easier to create really reproducible research documents in R.
}




One of R's main advantages is that it has a very active community of package developers who are rapidly expanding and improving its capabilities through user-created packages. Many pieces of research depend on these packages. Though R's active package development community enables this research, the fact that it is so active and packages can change very quickly means that it can be difficult to reproduce research that depends on particular versions. Therefore, for a piece of research to be reproducible, it is very important that a researcher fully cite--crucially including the version number--all of the R packages that their results depend on \citep[see][for more details]{Jackson2012}.\footnote{Of course it is also important to cite the version of R you are using the same reasons. This is especially true because researchers are likely relying on packages that come with R. This paper was written with R version 3.0.1.}

Though very important, we can do even better then just creating a static reference list of package citations. First to make our research really reproducible it should be reasonably easy for other researchers to not only understand how results were achieved, but also replicate those results. Literate programming technologies, such as \CRANpkg{knitr} \citep{R-knitr}, have made it much easier to reproduce research results. Would-be reproducers, can however face a problem when they compile documents created with these documents: they need to install any package dependencies not already in their library path and, they may need to individually hunt down specific versions of the packages used in the original paper.

Secondly, though we may intend to fully cite the versions of the packages we used in our research, we may fail to accurately do so. This is especially true if there are a number of packages that are updated over the course of the research process. If the package version cited differs from the one actually used and there has been a syntax or other change to the package between the two versions, then it could be very difficult to reproduce the research.

In this paper I propose and demonstrate a solution to these two problems. Researchers can use the \code{LoadandCite} command from the \CRANpkg{repmis} package\footnote{I would like to thank Karthik Ram, who contributed code to the command.} \citep{R-repmis} in their reproducible research documents to (a) load all of the packages that they used in the document, (b) create a BibTeX file with the full citation information of each package loaded and updates the file every time the paper is compiled, and (c) automatically install specific package versions. 

\section{Basic Syntax}

First let's look at the basic \code{LoadandCite} syntax. Below, we'll examine ways to include it in reproducible LaTeX documents. The main argument that LoadandCite takes is a character vector of package names. For example:

\begin{example}
  Packages <- c('car', 'knitr')
\end{example}

\noindent We can now simply add this vector object to \code{LoadandCite} and specify the name of the BibTeX file that we want to save the citations to.\footnote{If no file is specified, then the packages are only loaded.} Let's use the file name \file{ExamplePackage.bib}:

\begin{example}
  repmis::LoadandCite(pkgs = Packages, file = 'Example.bib')
\end{example}

\noindent The three packages will be loaded and a BibTeX file with citation information for the loaded version created in the working directory. \strong{Note} it is a good idea to use a file name that is different from the one you use for non-R package citations to avoid accidentally overwriting the file.

If we want to have the packages installed as well as loaded and cited we can set the argument \code{install = TRUE}. In addition to this we can include a character vector of package version numbers to have \code{LoadandCite} specify which versions of the packages to load from \href{http://cran.r-project.org/}{CRAN}. The order of the package versions must match the order of the versions listed in the \code{pkgs} argument. For example, to install \CRANpkg{car} \citep{R-car} version 2.0-17 and \CRANpkg{knitr} version 1.1 use:

\begin{example}
  Vers <- c('2.0-17', '1.1')

  repmis::LoadandCite(pkgs = Packages, install = TRUE,
                      versions = Vers, file = 'Example.bib')
\end{example}


\noindent You should avoid using old package versions in active research projects, so it is probably better to specify the versions only in documents you are making available for replication.\footnote{If \code{install = FALSE} (the default) then specific package versions will not be installed, i.e. \code{LoadandCite} will ignore any values in the argument \code{versions}.}

Finally, if you are reproducing a piece of research and installing old versions of packages it can be a good idea to install these into a separate library from the one that you normally use. You can specify what library to install the packages into with the \code{lib} argument.\footnote{If needed, remember to then add this library path to R with \code{.libPath}.}

\section{Using \code{LoadandCite} in a Knitted LaTeX Document}

Possibly the best way to use \pkg{LoadandCite} with your research is placing it in a `code chunk' at the beginning of a \CRANpkg{knitr} created LaTeX document\footnote{It would also work with a \CRANpkg{knitr} Markdown document rendered with \href{Pandoc}{http://johnmacfarlane.net/pandoc/}. This can be useful for creating reproducible webpages. See the \href{knitr documentation site}{http://yihui.name/knitr/demo/pandoc/} for how to use Pandoc from R.} and using it to load every R package used to create the document. For example:

\begin{example}
\textless{\textless}include=FALSE\textgreater{\textgreater}=
  Packages <- c('car', 'knitr')
  repmis::LoadandCite(pkgs = Packages, file = 'Example.bib')
@
\end{example}

\noindent The code \code{\textless{\textless}include=FALSE\textgreater{\textgreater}=} and \code{@} delimit the beginning and ending of the code chunk. The \code{include=FALSE} argument tells \CRANpkg{knitr} to not include any messages, warnings, and so on created by the code in the compiled presentation document.

You can then cite the packages using BibTeX cite keys as usual. The BibTeX keys created by \code{LoadandCite}\footnote{The keys generated with code based on \CRANpkg{knitr}'s \code{write\_bib} command. \code{LoadandCite} does not formally depend on \CRANpkg{knitr} to make it possible to install old versions of that package.} follow the following formula \code{R-PACKAGE\_NAME}. For example, the \CRANpkg{car} cite key will be \code{R-car}. To create a citation like \citep{R-car} simply type: \verb|\citep{R-car}| in the text of your LaTeX document where you would like the citation to appear. Additionally add the name of the BibTeX file created by \code{LoadandCite} to your \code{bibliography} command near the end of your LaTeX document, i.e. \verb|\bibliography{Example}|.

Note that because \CRANpkg{repmis} is itself an add on R package, you should clearly inform any body who would want to reproduce your research that they need to install it first before running \code{LoadandCite}. You could do this in a README file accompanying the replication files.

\section{Summary}

R's capabilities are developing very quickly. To be able to take advantage of these expanding capabilities while also making our research easily reproducible in the future, we need to make sure that the add-on packages we use are fully cited and that it is easy for future users to easily install the versions we used. Using \code{LoadandCite} in dynamically created research documents helps us do this. 

\bibliography{Gandrud,GandrudPackages}

\address{Christopher Gandrud\\
  Yonsei University\\
  1 Yonseidae-Gil, Wonju \\
  Gangwon-do, 220-710\\
  Republic of Korea}
\email{christopher.gandrud@gmail.com}

\end{article}

\end{document}
